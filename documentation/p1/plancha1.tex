\documentclass[11pt]{article}
\usepackage[a4paper, margin=1.5cm]{geometry}
\usepackage[utf8]{inputenc}
\usepackage{babel}
\usepackage[spanish]{layout}
\usepackage[article]{ragged2e}
\usepackage{textcomp}
\usepackage{amsmath}
\usepackage{amssymb}
\usepackage{amsfonts}
\usepackage{proof}
\usepackage{enumerate}
\usepackage{graphicx}
\usepackage{multirow}

\setlength{\parindent}{0pt}

\title{
    Entrega 3 \\
    \large Sistemas Operativos II}
\author{Mellino, Natalia \and Farizano, Juan Ignacio}
\date{}

\begin{document}
\maketitle

\noindent\rule{\textwidth}{1pt}

\begin{enumerate}
    \item Se prefiere emular una CPU porque de esta forma se simplifica mucho
          el desarrollo de sistemas operativos al reducir el ciclo
          compilar-ejecutar-depurar y permite el uso de depuradores ya disponibles.
    \item La memoria se define como una constante en el archivo \texttt{/machine/mmu.hh} como 
          \texttt{MEMORY\_SIZE = NUM\_PHYS\_PAGES x PAGE\_SIZE}. Como tenemos 32 páginas
          y cada una tiene un tamaño de 128 bytes el tamaño de la memoria es 32 x 128 = 4096 bytes.
    \item  Modificaría la cantidad de páginas. El tamaño de cada página se deja fijo en 128 bytes para 
           que coincida con el tamaño del sector del disco y proveernos así más simplicidad.
    \item El disco cuenta con una única superficie dividida en pistas y a su vez, cada pista está dividida 
          en sectores. Por defecto, el disco de la máquina de Nachos cuenta con 32 pistas y 32 sectores por 
          cada pista. Luego el tamaño de cada sector, como fue mencionado en el punto anterior, es de 128 bytes. 
          Como resultado contamos con 32 x 32 x 128 = 131.072 bytes.
    \item En el archivo \texttt{/machine/encoding.hh} podemos encontrar todas las instrucciones enumeradas, las 
          instrucciones \texttt{OP\_UNIMP} y \texttt{OP\_RES} no son  simuladas, por lo tanto contamos con un total de 61 instrucciones 
          MIPS  que simula la máquina virtual.
    \item Utilizando el comando \texttt{grep} podemos ver que la función main está
          definida en los siguientes archivos:
    \begin{itemize}
        \item \texttt{/bin/fuse/nachosfuse.c}
        \item \texttt{/bin/out.c}
        \item \texttt{/bin/coff2noff.c}
        \item \texttt{/bin/disasm.c}
        \item \texttt{/bin/readnoff.c}
        \item \texttt{/bin/main.c}
        \item \texttt{/bin/coff2flat.c}
        \item \texttt{/userland/echo.c}
        \item \texttt{/userland/filetest.c}
        \item \texttt{/userland/tiny\_shell.c}
        \item \texttt{/userland/matmult.c}
        \item \texttt{/userland/touch.c}
        \item \texttt{/userland/halt.c}
        \item \texttt{/userland/sort.c}
        \item \texttt{/userland/shell.c}
        \item \texttt{/threads/main.c}
    \end{itemize}
    Al correr make, en el archivo Makefile.depends, podemos ver que la función
    main se encuentra en el archivo \texttt{/threads/main.cc}
    \item
    \begin{itemize}
    \item \texttt{Initialize} en \texttt{/threads/system.cc}
        \begin{itemize}
            \item \texttt{ASSERT} en \texttt{/lib/assert.hh}
            \item \texttt{ParseDebugOpts} en \texttt{/threads/system.cc}
            \item \texttt{RandomInit} en \texttt{/machine/system\_dep.cc}
            \item \texttt{SetFlags} en \texttt{/lib/debug.cc}
            \item \texttt{SetOpts} en \texttt{/lib/debug.cc}
            \item \texttt{TimerInterruptHandler} en \texttt{/threads/system.cc}
            \item \texttt{SetStatus} en \texttt{/threads/threads.cc}
            \item \texttt{Enable} en \texttt{/machine/interrupt.cc}
            \item \texttt{CallOnUserAbort} en \texttt{/machine/system\_dep.cc}
            \item \texttt{Cleanup} en \texttt{/threads/system.cc}
            \item \texttt{SetUp} en \texttt{/threads/preemptive.cc}
            \item \texttt{SetExceptionHandlers} en \texttt{/userprog/exception.cc}
        \end{itemize}
    \item \texttt{DEBUG} en \texttt{/lib/utility.hh}
        \begin{itemize}
            \item \texttt{Print} en \texttt{/lib/debug.cc}
        \end{itemize}
    \item \texttt{SysInfo} en \texttt{/threads/sys\_info.cc}
    \item \texttt{PrintVersion} en \texttt{/threads/main.cc}
    \item \texttt{ThreadTest} en \texttt{/threads/thread\_test.cc}
        \begin{itemize}
            \item \texttt{DEBUG} en \texttt{/lib/utily.hh}
            \item \texttt{Choose} en \texttt{/threads/thread\_test.cc}
            \item \texttt{Run} en \texttt{/threads/thread\_test.cc}
        \end{itemize}
    \item \texttt{Halt} en \texttt{/machine/interrupt.cc}
        \begin{itemize}
            \item \texttt{Print} en \texttt{/machine/statistics.cc}
            \item \texttt{Cleanup} en \texttt{/threads/system.cc}
        \end{itemize}
    \item \texttt{ASSERT} en \texttt{/lib/assert.hh}
        \begin{itemize}
            \item \texttt{Assert} en \texttt{/lib/assert.cc}
        \end{itemize}
    \item \texttt{StartProcess} en \texttt{/userprog/prog\_test.cc}
        \begin{itemize}
            \item \texttt{ASSERT} en \texttt{/lib/assert.hh}
            \item \texttt{Open} en \texttt{/filesys/file\_system.cc}
            \item \texttt{InitRegisters} en \texttt{/userprog/address\_space.cc}
            \item \texttt{RestoreState} en \texttt{/userprog/address\_space.cc}
            \item \texttt{Run} en \texttt{/userprog/debugger\_command\_manager.cc}
        \end{itemize}
    \item \texttt{ConsoleTest} en \texttt{/userprog/prog\_test.cc}
    \item 
        \begin{itemize}
            \item \texttt{P} en \texttt{/threads/semaphore.cc}
            \item \texttt{GetChar} en \texttt{/machine/console.cc}
            \item \texttt{PutChar} en \texttt{/machine/console.cc}
        \end{itemize}
    \item \texttt{Copy} en \texttt{/filesys/fs\_test.cc}
        \begin{itemize}
            \item \texttt{ASSERT} en \texttt{/lib/assert.hh}
            \item \texttt{DEBUG} en \texttt{/lib/utility.hh}
            \item \texttt{Create} en \texttt{/filesys/file\_system.cc}
            \item \texttt{Open} en \texttt{/filesys/file\_system.cc}
            \item \texttt{Write} en \texttt{/filesys/open\_file.cc}
        \end{itemize}
    \item \texttt{Print} en \texttt{/threads/scheduler.cc}
        \begin{itemize}
            \item \texttt{Apply} en \texttt{/lib/list.hh}
            \item \texttt{ThreadPrint} en \texttt{/threads/scheduler.cc}
        \end{itemize}
    \item \texttt{Remove} en \texttt{/lib/list.hh}
    \item \texttt{List} en \texttt{/filesys/directory.cc}
    \item \texttt{Check} en \texttt{/filesys/file\_system.cc}
        \begin{itemize}
            \item \texttt{DEBUG} en \texttt{/lib/utility.hh}
            \item \texttt{Mark} en \texttt{/lib/bitmap.cc}
            \item \texttt{GetRaw} en \texttt{/filesys/file\_header.cc}
            \item \texttt{FetchFrom} en \texttt{/filesys/file\_header.cc}
            \item \texttt{CheckForError} en \texttt{/filesys/file\_system.cc}
            \item \texttt{CheckFileHeader} en \texttt{/filesys/file\_system.cc}
            \item \texttt{CheckBitmaps} en \texttt{/filesys/file\_system.cc}
        \end{itemize}
    \item \texttt{PerformanceTest} en \texttt{/filesys/fs\_test.cc}
        \begin{itemize}
            \item \texttt{Print} en \texttt{/filesys/fs\_test.cc}
            \item \texttt{FileWrite} en \texttt{/filesys/fs\_test.cc}
            \item \texttt{FileRead} en \texttt{/filesys/fs\_test.cc}
            \item \texttt{Remove} en \texttt{/lib/list.hh}
        \end{itemize}
    \item \texttt{Delay} en \texttt{/machine/system\_dep.cc}
    \item \texttt{MailTest} en \texttt{/network/net\_test.cc}
        \begin{itemize}
            \item \texttt{Send} en \texttt{/network/post.cc}
            \item \texttt{Receive} en \texttt{/network/post.cc}
            \item \texttt{Halt} en \texttt{/machine/interrupt.cc}
        \end{itemize}
    \item \texttt{Finish} en \texttt{/threads/thread.cc}
        \begin{itemize}
            \item \texttt{SetLevel} en \texttt{/machine/interrupt.cc}
            \item \texttt{ASSERT} en \texttt{/lib/utility.hh}
            \item \texttt{DEBUG} en \texttt{/lib/utility.hh}
            \item \texttt{GetName} en \texttt{/threads/thread.cc}
            \item \texttt{Sleep} en \texttt{/threads/thread.cc}
        \end{itemize}
    \end{itemize}
    \item
        \begin{itemize}
            \item \texttt{ASSERT}: aborta el program en caso de que una condición
                  no se cumpla.
            \item \texttt{DEBUG}: muestra por pantalla mensajes de depuración 
                  brindando al usuario flags de depuración y permitiendole definir las suyas propias.
        \end{itemize}
    \item
        \begin{itemize}
            \item \; \texttt{+} \;-\; encender todas las banderas de depuración
            \item \; \texttt{t} \;-\; mensajes sobre sistemas de threads
            \item \; \texttt{s} \;-\; mensajes sobre semaforos, locks y condiciones
            \item \; \texttt{i} \;-\; mensajes acerca de emulación de interrupciones
            \item \; \texttt{m} \;-\; mensajes sobre emulación de la máquina
            \item \; \texttt{d} \;-\; mensajes sobre emulación del disco
            \item \; \texttt{f} \;-\; mensajes sobre el sistema de archivos
            \item \; \texttt{a} \;-\; mensajes sobre el espacio de direcciones
            \item \; \texttt{e} \;-\; mensajes sobre manejo de excepciones
            \item \; \texttt{n} \;-\; mensajes sobre emulación de red
        \end{itemize}
    \item
        \begin{itemize}
            \item \texttt{USER\_PROGRAM} - definida en los makefile de userprog, vmem, network y filesys
            \item \texttt{FILESYS\_NEEDED} - definida en los makefile de filesys, network, userprog y vmem
            \item \texttt{FILESYS\_STUB} - definida en userprog y vmem
            \item \texttt{NETWORK} - definida en network
        \end{itemize}
    \item
        \begin{itemize}
            \item Opciones generales
                \begin{itemize}
                    \item \texttt{-d} -- hace que ciertos mensajes de depuración se muestren
                    \item \texttt{-do} -- habilita opciones que modifican el comportamiento cuando se imprimen mensajes de depuración
                    \item \texttt{-p} -- habilita multitarea preemtiva para hilos del kernel
                    \item \texttt{-rs} -- hace que ocurra 'yield' en lugares aleatorios
                    \item \texttt{-z} -- imprime información sobre la versión y copyright y sale
                \end{itemize}
            \item Opciones de hilos
                \begin{itemize}
                    \item \texttt{-tt} -- prueba el subsistema de threads, se le pide al usuario 
                          elegir un test a correr de entre una colección de tests disponibles
                \end{itemize}
            \item Opciones de \texttt{USER\_PROGRAM}
                \begin{itemize}
                    \item \texttt{-s} -- causa que el programa se ejecute en modo de un solo paso
                    \item \texttt{x} -- corre un programa de usuario
                    \item \texttt{-tc} -- prueba la consola
                \end{itemize}
            \item Opciones del sistema de archivos
                \begin{itemize}
                    \item \texttt{-f} -- causa que el disco físico sea formateado
                    \item \texttt{-cp} -- copia un archivo de UNIX a nachos
                    \item \texttt{-pr} -- imprime un archivo de nachos por la salida standard
                    \item \texttt{rm} -- elimina un archivo nachos del sistema de archivos
                    \item \texttt{ls} -- lista los contenidos del directorio de nachos
                    \item \texttt{-D} -- imprime contenidos del sistema de archivos entero
                    \item \texttt{-c} -- verifica la integridad del sistema de archivos
                    \item \texttt{-tf} -- testea la performance del sistema de archivos de nachos
                \end{itemize}
            \item Opciones de red
                \begin{itemize}
                    \item \texttt{-n} -- setea la confiabilidad de la red
                    \item \texttt{-id} -- setea el host id de esta maquinan (necesario para la red)
                    \item \texttt{-tn} -- corre un simple test del software de red de nachos  
                \end{itemize}
        \end{itemize}

        \addtocounter{enumi}{1} % Suma al contador de enumerate.
        
        \item La primera es una lista enlazada simple donde el acceso de los
              hilos no tiene restricciones, en cambio en la clase SynchList, si la
              lista está vacía y un hilo quiere remover un elemento de ella, deberá
              esperar a que contenga un elemento. Además, sólo un thread a la vez
              puede acceder a las estructuras de datos de lalista.
        \addtocounter{enumi}{3}
        \item De forma análoga al problema de la entrega 1, se soluciona realizando
              el yield recién luego de incrementar el contador.
\end{enumerate}

\end{document}